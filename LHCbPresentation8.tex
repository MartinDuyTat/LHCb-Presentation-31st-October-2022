%%%%%%%%%%%%%%%%%%%%%%%%%%%%%%%%%%%%%%%%%%%%%%%%%%%%%%%%%%%%%%%%%%%%%%
% Overleaf (WriteLaTeX) Example: Molecular Chemistry Presentation
%
% Source: http://www.overleaf.com
%
% In these slides we show how Overleaf can be used with standard 
% chemistry packages to easily create professional presentations.
% 
% Feel free to distribute this example, but please keep the referral
% to overleaf.com
% 
%%%%%%%%%%%%%%%%%%%%%%%%%%%%%%%%%%%%%%%%%%%%%%%%%%%%%%%%%%%%%%%%%%%%%%

\documentclass{beamer}

\mode<presentation>
{
  \usetheme{Madrid}       % or try default, Darmstadt, Warsaw, ...
  \usecolortheme{default} % or try albatross, beaver, crane, ...
  \usefonttheme{default}    % or try default, structurebold, ...
  \setbeamertemplate{navigation symbols}{}
  \setbeamertemplate{caption}[numbered]
} 

\usepackage[english]{babel}
\usepackage[utf8x]{inputenc}
\usepackage{graphicx}
\usepackage{hyperref}
  \hypersetup{colorlinks=true}
  \hypersetup{urlcolor=blue}
  \hypersetup{linkcolor = .}
\usepackage{xcolor}
\usepackage{siunitx}
  \sisetup{separate-uncertainty = true}
\usepackage{physics}
\usepackage[font=small,labelfont=bf,justification=centering]{caption}
\usepackage{subcaption}
\usepackage[en-GB]{datetime2}
\usepackage{overpic}
\usepackage{feynmp}
\DeclareGraphicsRule{*}{mps}{*}{}
\usepackage{scalerel}
\newcommand{\mylbrace}[2]{\vspace{#2pt}\hspace{6pt}\scaleleftright[\dimexpr5pt+#1\dimexpr0.06pt]{\lbrace}{\rule[\dimexpr2pt-#1\dimexpr0.5pt]{-4pt}{#1pt}}{.}}
\newcommand{\myrbrace}[2]{\vspace{#2pt}\scaleleftright[\dimexpr5pt+#1\dimexpr0.06pt]{.}{\rule[\dimexpr2pt-#1\dimexpr0.5pt]{-4pt}{#1pt}}{\rbrace}\hspace{6pt}}
\usepackage{ulem} % Line across text

% Here's where the presentation starts, with the info for the title slide
\title[ARC]{ARC: A RICH detector proposal for FCC-ee}

\author{Martin Tat}
\institute{Oxford LHCb}
\date{31st October 2022}

\titlegraphic{\includegraphics[height = 2cm]{lhcb.jpg}\hspace{1cm}~%
              \includegraphics[height = 2cm]{OxfordLogo.pdf}\hspace{1cm}~%
              \includegraphics[height = 2cm]{bes3.jpg}}

\begin{document}

\begin{frame}
  \titlepage
\end{frame}

% These three lines create an automatically generated table of contents.
%\begin{frame}{Outline}
%  \tableofcontents
%\end{frame}

\section{Analysis work summary}

\begin{frame}{Analysis work summary}
  \begin{itemize}
    \setlength\itemsep{0.5em}
    \item{For my PhD thesis I'm working on two related projects:}
    \begin{itemize}
      \setlength\itemsep{0.5em}
      \item{Measurement of $D^0\to K^+K^-\pi^+\pi^-$ strong phases $c_i$ and $s_i$ at BESIII}
      \item{Binned GGSZ analysis of $\gamma$ in $B^\pm\to Dh^\pm$, $D\to K^+K^-\pi^+\pi^-$ at LHCb}
    \end{itemize}
    \item{Over the summer I have worked on two papers:}
  \end{itemize}
  \begin{figure}
    \begin{subfigure}{0.35\textwidth}
      \includegraphics[width = 1.0\textwidth, page = 1]{Plots/BESIII_PAPER_DocDB_1111_NoAuthorList.pdf}
    \end{subfigure}%
    \hspace{1cm}
    \begin{subfigure}{0.35\textwidth}
      \includegraphics[width = 1.0\textwidth, page = 1]{Plots/LHCb-PAPER-2022-037-v3.pdf}
    \end{subfigure}
  \end{figure}
\end{frame}

\begin{frame}{Analysis work summary}
  \begin{itemize}
    \item{BESIII paper: Measurement of the CP-even fraction of $D^0\to K^+K^-\pi^+\pi^-$}
    \begin{itemize}
      \item{Has gone through all BESIII reviews, except for Spokesperson approval}
      \item{We were unfortunately assigned the spokesperson with a backlog of 5 papers, expect 2-3 months delay...}
    \end{itemize}
  \end{itemize}
  \begin{figure}
    \includegraphics[width = 0.55\textwidth, page = 1]{Plots/CPeven_fraction_combination_CPtags.png}
    \caption{CP-even and CP-odd branching fractions of $D\to K^+K^-\pi^+\pi^-$}
  \end{figure}
\end{frame}

\begin{frame}{Analysis work summary}
  \begin{itemize}
    \item{LHCb paper: A study of CP violation in the decays $B^\pm\to[K^+K^-\pi^+\pi^-]_Dh^\pm$ $(h = K, \pi)$ and $B^\pm\to[\pi^+\pi^-\pi^+\pi^-]_Dh^\pm$}
    \begin{itemize}
      \item{Binned model dependent GGSZ analysis of $B^\pm\to[K^+K^-\pi^+\pi^-]_Dh^\pm$}
      \item{Phase space integrated GLW analysis of $B^\pm\to[h^+h^-\pi^+\pi^-]_Dh^\pm$}
      \item{Paper has just been through 1st collaboration wide circulation, currently waiting for RC and EB to review the new draft}
    \end{itemize}
  \end{itemize}
  \begin{figure}
    \includegraphics[width = 0.5\textwidth, page = 1]{Plots/BinAsymmetries_dk.png}
    \caption{$B^\pm\to[K^+K^-\pi^+\pi^-]_DK^\pm$ bin asymmetries}
  \end{figure}
\end{frame}

\section{ARC}

\begin{frame}{ARC: A solution for PID at FCC-ee}
  \begin{center}
    \huge ARC: A solution for PID at FCC-ee\\
    \Large\textbf{A}rray of \textbf{R}ICH \textbf{C}ells\\~\\
    \large A side project I've worked on while my BESIII and LHCb\\papers go through review
  \end{center}
  \begin{figure}
    \centering
    \begin{subfigure}{0.4\textwidth}
      \includegraphics[width = 1.0\textwidth]{Plots/ARC_Cell.png}
    \end{subfigure}%
    \hspace{1cm}
    \begin{subfigure}{0.3\textwidth}
      \includegraphics[width = 1.0\textwidth]{Plots/CompoundEyes.jpg}
    \end{subfigure}
    \caption{ARC: Array of RICH Cells}
  \end{figure}
\end{frame}

\begin{frame}{RICH detectors}
  \begin{itemize}
    \setlength\itemsep{0.5em}
    \item{Excellent hadron PID is crucial in flavour physics}
    \item{RICH detectors are very powerful for particle ID at high momentum}
    \item{At LHCb, $\pi$-$K$ separation is excellent up to $\SI{100}{\giga\eV}$}
    \item{A $4\pi$ collider RICH layout was previously used at DELPHI and SLD}
    \begin{itemize}
      \item{Challenging because of the space required}
    \end{itemize}
  \end{itemize}
  \begin{figure}
    \centering
    \vspace{-0.2cm}
    \begin{subfigure}{0.35\textwidth}
      \includegraphics[height = 3.0cm]{Plots/DELPHI_RICH.jpg}
      \caption{DELPHI RICH layout}
    \end{subfigure}%
    \begin{subfigure}{0.35\textwidth}
      \includegraphics[height = 3.0cm]{Plots/KandPi_2_K.pdf}
      \caption{LHCb RICH performance}
    \end{subfigure}%
    \begin{subfigure}{0.3\textwidth}
      \vspace{-1cm}
      \includegraphics[height = 4.0cm]{Plots/LHCb_RICH.png}
      \caption{LHCb RICH layout}
    \end{subfigure}
  \end{figure}
\end{frame}

\begin{frame}{Motivation for RICH at FCC-ee}
  \begin{itemize}
    \setlength\itemsep{0.7em}
    \item{FCC-ee will collect $5\times 10^{12}$ $Z$ boson decays in $4$ years}
    \begin{itemize}
      \item{Allows for a world-leading flavour physics programme}
      \item{Combined with excellent PID capabilities, FCC-ee will reach an unprecedented precision}
    \end{itemize}
    \item{Good PID performance is also required for Higgs, $WW$ and $t\bar{t}$ physics}
    \begin{itemize}
      \item{In particular, kaon ID is crucial for $H\to s\bar{s}$}
    \end{itemize}
  \end{itemize}
  \begin{figure}
    \centering
    \vspace{-0.2cm}
    \begin{subfigure}{0.5\textwidth}
      \includegraphics[height = 4cm, trim = {0 0 22.5cm 0}, clip = true]{Plots/p_spectrum_crop.png}
    \end{subfigure}%
    \begin{subfigure}{0.5\textwidth}
      \includegraphics[height = 4cm, trim = {22.0cm 0 0 0}, clip = true]{Plots/p_spectrum_crop.png}
    \end{subfigure}
    \caption{$B_s^0\to D_s^\pm K^\mp$\\$B$ physics requires pion-kaon separation from low momentum up to $\SI{40}{\giga\eV}$}
  \end{figure}
\end{frame}

\begin{frame}{Motivation for RICH at FCC-ee}
  \begin{itemize}
    \setlength\itemsep{0.7em}
    \item{FCC-ee will collect $5\times 10^{12}$ $Z$ boson decays in $4$ years}
    \begin{itemize}
      \item{Allows for a world-leading flavour physics programme}
      \item{Combined with excellent PID capabilities, FCC-ee will reach an unprecedented precision}
    \end{itemize}
    \item{Good PID performance is also required for Higgs, $WW$ and $t\bar{t}$ physics}
    \begin{itemize}
      \item{In particular, kaon ID is crucial for $H\to s\bar{s}$}
    \end{itemize}
  \end{itemize}
  \begin{figure}
    \centering
    \vspace{-0.2cm}
    \includegraphics[height = 4cm]{Plots/LHCb_BsDsK.pdf}
    \caption{$B_s^0\to D_s^\pm K^\mp$\\The $B_s^0\to D_s^\pm\pi^\mp$ background would be $10$ times larger without PID capabilities!}
  \end{figure}
\end{frame}

\begin{frame}{\textbf{A}rray of \textbf{R}ICH \textbf{C}ells}
  \begin{itemize}
    \setlength\itemsep{0.2em}
    \item{\textbf{A}rray of \textbf{R}ICH \textbf{C}ells (ARC): A novel RICH detector concept}
    \begin{itemize}
      \item{First presented by R. Forty at \href{https://indico.cern.ch/event/995850/contributions/4406336/attachments/2274813/3864163/ARC-presentation.pdf}{FCC week 2021}}
      \item{Compact, low-mass solution for particle ID for FCC-ee}
      \item{Concept inspired by the compound eyes of an insect}
    \end{itemize}
    \item{Adapted to fit into the \href{https://arxiv.org/abs/1911.12230}{CLD experiment} concept, taking $10\%$ from the tracker volume}
    \begin{itemize}
      \item{Radial depth of $\SI{20}{\centi\meter}$, radius of $\SI{2.1}{\meter}$ and a length of $\SI{4.4}{\meter}$}
      \item{Aim to keep material budget below $0.1X_0$}
    \end{itemize}
    \item{Aerogel and gas radiators with a spherical mirror}
    \begin{itemize}
      \item{Aerogel also acts as thermal insulation between gas and detector}
    \end{itemize}
  \end{itemize}
  \begin{figure}
    \centering
    \begin{subfigure}{0.4\textwidth}
      \includegraphics[width = 1.0\textwidth]{Plots/ARC_Cell.png}
    \end{subfigure}%
    \hspace{1cm}
    \begin{subfigure}{0.3\textwidth}
      \includegraphics[width = 1.0\textwidth]{Plots/CompoundEyes.jpg}
    \end{subfigure}
    \caption{ARC has a cellular structure, similar to an insect's compound eyes}
  \end{figure}
\end{frame}

\begin{frame}{\textbf{A}rray of \textbf{R}ICH \textbf{C}ells}
  \begin{itemize}
    \setlength\itemsep{0.5em}
    \item{All cells are the same size, organised on a hexagonal grid}
    \begin{itemize}
      \item{Barrel (endcap) has $945$ ($384$) cells in total, where $18$ ($21$) are unique}
      \item{Hexagonal shape avoids the corners, where performance is worse}
    \end{itemize}
  \end{itemize}
  \begin{figure}
    \centering
    \begin{subfigure}{0.6\textwidth}
      \includegraphics[width = 1.0\textwidth]{Plots/BarrelCells.png}
    \end{subfigure}%
    \begin{subfigure}{0.3\textwidth}
      \includegraphics[width = 1.0\textwidth]{Plots/EndcapCells.png}
    \end{subfigure}
    \caption{Barrel (left) and endcap (right) cells}
  \end{figure}
\end{frame}

\begin{frame}{ARC radiators}
  \begin{itemize}
    \setlength\itemsep{0.1em}
    \item{$C_4F_{10}$:}
    \begin{itemize}
      \setlength\itemsep{0.2em}
      \item{Baseline assumption, well known from LHCb RICH1}
      \item{$n = 1.0014\implies\theta_c = \SI{53}{\milli\radian}$, suitable for high momentum particles}
      \item{C$_4$F$_{10}$ is a greenhouse gas, plan to replace with suitable Novec gas, such as C$_5$F$_{10}$O}
    \end{itemize}
    \item{Aerogel:}
    \begin{itemize}
      \setlength\itemsep{0.2em}
      \item{Well known as a RICH radiator, e.g. from ARICH at Belle II}
      \item{$n = 1.01$-$1.10\implies\theta_c = 141$-$\SI{430}{\milli\radian}$, suitable at low momentum}
      \item{Very low thermal conductivity}
      \begin{itemize}
        \item{Suitable to separate gas from detector, which must be cooled}
        \item{Cherenkov photons come for ``free'' and are focused by the same mirror}
      \end{itemize}
      \item{Drawback: Some loss of photons from scattering}
    \end{itemize}
  \end{itemize}
  \begin{figure}
    \centering
    \begin{subfigure}{0.25\textwidth}
      \includegraphics[width = 1.0\textwidth]{Plots/BelleAerogelTiles.png}
    \end{subfigure}%
    \hspace{0.2cm}
    \begin{subfigure}{0.25\textwidth}
      \includegraphics[width = 1.0\textwidth]{Plots/AerogelTransmission.png}
    \end{subfigure}
    \caption{Belle aerogel tiles (left) and aerogel transmission function (right).}
  \end{figure}
\end{frame}

\begin{frame}{Photon hits}
  \begin{figure}
    \centering
    \includegraphics[width = 0.65\textwidth, trim = {2cm 9cm 3cm 6cm}, clip = true]{Plots/Display1.pdf}
    \caption{Photon hits on photodetector}
  \end{figure}
\end{frame}

\begin{frame}{Event display}
  \begin{figure}
    \centering
    \includegraphics[width = 0.85\textwidth, trim = {0cm 2cm 0cm 0cm}, clip = true]{Plots/Display2.pdf}
    \caption{$B_s\to D_sK$ (no magnetic field yet)}
  \end{figure}
\end{frame}

\begin{frame}{Optimisation of ARC layout}
  \begin{itemize}
    \setlength\itemsep{0.7em}
    \item{The following procedure is used to evaluate the ARC performance:}
    \begin{enumerate}
      \setlength\itemsep{0.2em}
      \item{Generate straight particle track from IP and trace it through ARC}
      \item{Generate Cherenkov photons from gas radiator}
      \item{Track photons through the optics and to detector}
      \item{Reconstruct Cherenkov angles and calculate standard deviation}
    \end{enumerate}

    \item{Three sources of uncertainty are considered:}
    \begin{enumerate}
      \setlength\itemsep{0.2em}
      \item{Emission point uncertainty: Emission point is assumed to be the mid-point of the track inside the gaseous radiator}
      \item{Chromatic dispersion uncertainty: Spread in Cherenkov angle due to wavelength dependence on refractive index}
      \item{Pixel size: Will be chosen so that it does not limit the performance}
    \end{enumerate}
  \end{itemize}
  \begin{block}{Minimise the Cherenkov angle uncertainty:}
    \begin{equation*}
      \Delta\theta = \frac{1}{\sqrt{N}}\times\frac{1}{1 - N}\times\sum_{i = 0}^{N - 1}(\theta - \bar{\theta})^2
    \end{equation*}
  \end{block}
\end{frame}

\begin{frame}{Examples of photon tracking through optimised layout}
  \begin{figure}
    \centering
    \begin{subfigure}{0.45\textwidth}
      \includegraphics[width = 1.0\textwidth, trim = {11cm 5cm 3.5cm 0}, clip = true]{Plots/EventDisplay_MainRow.pdf}
    \end{subfigure}%
    \hspace{0.2cm}
    \begin{subfigure}{0.45\textwidth}
      \includegraphics[width = 1.0\textwidth, trim = {11.6cm 5cm 2.9cm 0}, clip = true]{Plots/EventDisplay_MainRow_Aerogel.pdf}
    \end{subfigure}
    \caption{Tracking of photons from gas radiator (left) and aerogel radiator (right) through the ARC optics}
  \end{figure}
  \vspace{-0.3cm}
  \begin{itemize}
    \item{Parameters that are optimised:}
    \begin{itemize}
      \item{Mirror curvature}
      \item{Mirror vertical and horizontal position}
      \item{Detector horizontal position and tilt}
    \end{itemize}
  \end{itemize}
\end{frame}

\begin{frame}{Technical details about minimisation}
  \setlength\itemsep{1.0em}
  \begin{itemize}
    \item{Total Cherenkov angle uncertainty $\sigma(\vec{x})$ is not easy to calculate analytically}
    \begin{itemize}
      \item{$\vec{x}$ are the 5 parameters we want to optimise}
    \end{itemize}
    \item{Finite number of photons $\implies\sigma(\vec{x})$ is not differentiable}
    \begin{itemize}
      \item{Cannot be minimised using conventional methods (Minuit, etc)}
    \end{itemize}
    \item{I have experimented with a new type of minimisation algorithms: Stochastic optimisation}
    \begin{itemize}
      \item{\href{https://en.wikipedia.org/wiki/Differential_evolution}{Differential evolution}}
      \item{Start with a population of possible solutions, form new solutions by combining (mutating) existing solutions}
      \item{Advantage: Doesn't require initial guess, robust against functions that a not continuous, noisy, change over time, etc}
      \item{Disadvantage: No way to tell if optimal solution has been found, so it requires many iterations}
    \end{itemize}
  \end{itemize}
\end{frame}

\begin{frame}{Cherenkov angle uncertainty for gas radiator}
  \begin{figure}
    \centering
    \vspace{-0.2cm}
    \begin{subfigure}{0.35\textwidth}
      \includegraphics[width = 1.0\textwidth]{Plots/NumberDetectedPhotons_Barrel_Gas.png}
      \vspace{-0.75cm}
      \caption{Mean number of photons detected}
    \end{subfigure}
    \begin{subfigure}{0.35\textwidth}
      \includegraphics[width = 1.0\textwidth]{Plots/SinglePhotonCherenkovAngles_Barrel_Gas.png}
      \vspace{-0.75cm}
      \caption{Single photon uncertainty:\\ $\SI{1.0}{\milli\radian}$}
    \end{subfigure}%
    \begin{subfigure}{0.35\textwidth}
      \includegraphics[width = 1.0\textwidth]{Plots/TotalCherenkovUncertainty_Barrel_Gas.png}
      \vspace{-0.75cm}
      \caption{Total uncertainty:\\ $\SI{0.4}{\milli\radian}$}
    \end{subfigure}
    \vspace{-0.1cm}
    \caption{Gas radiator performance averaged over all barrel cells}
  \end{figure}
\end{frame}

\begin{frame}{Cherenkov angle uncertainty for aerogel radiator}
  \begin{figure}
    \centering
    \vspace{-0.2cm}
    \begin{subfigure}{0.35\textwidth}
      \includegraphics[width = 1.0\textwidth]{Plots/NumberDetectedPhotons_Barrel_Aerogel.png}
      \vspace{-0.75cm}
      \caption{Mean number of photons detected}
    \end{subfigure}
    \begin{subfigure}{0.35\textwidth}
      \includegraphics[width = 1.0\textwidth]{Plots/SinglePhotonCherenkovAngles_Barrel_Aerogel.png}
      \vspace{-0.75cm}
      \caption{Single photon uncertainty:\\ $\SI{2.5}{\milli\radian}$}
    \end{subfigure}%
    \begin{subfigure}{0.35\textwidth}
      \includegraphics[width = 1.0\textwidth]{Plots/TotalCherenkovUncertainty_Barrel_Aerogel.png}
      \vspace{-0.75cm}
      \caption{Total uncertainty:\\ $\SI{0.9}{\milli\radian}$}
    \end{subfigure}
    \vspace{-0.1cm}
    \caption{Aerogel radiator performance averaged over all barrel cells}
  \end{figure}
\end{frame}

\begin{frame}{Performance of optimised ARC}
  \begin{figure}
    \centering
    \vspace{-0.2cm}
    \begin{subfigure}{0.5\textwidth}
      \includegraphics[width = 1.0\textwidth]{Plots/Significance_Scatter_PionKaon.png}
    \end{subfigure}%
    \begin{subfigure}{0.5\textwidth}
      \includegraphics[width = 1.0\textwidth]{Plots/Significance_Scatter_ProtonKaon.png}
    \end{subfigure}
    \vspace{-0.4cm}
    \caption{Separation significance per track for $\pi$-$K$ (left) and $p$-$K$ (right)}
  \end{figure}
  \vspace{-0.4cm}
  \begin{itemize}
    \setlength\itemsep{0.0em}
    \item{Gas (aerogel) provides over $3\sigma$ pion-kaon separation in the range $10$-$\SI{50}{\giga\eV}$ ($2$-$\SI{10}{\giga\eV}$)}
    \begin{itemize}
      \item{Effect of magnetic field not yet included in these studies}
    \end{itemize}
    \item{Combined, the aerogel and gas ensure excellent PID performance over the whole range of interest to flavour physics}
  \end{itemize}
\end{frame}

\begin{frame}{Summary and next steps}
  \begin{itemize}
    \setlength\itemsep{1.0em}
    \item{ARC is a low mass and compact cellular PID detector designed to occupy minimum space ($\SI{20}{\centi\meter}$ in the radial dimension) in a $4\pi$ detector at an $e^+e^-$ collider such as FCC-ee}
    \item{We have developed an optimised layout that should achieve a $3\sigma$ kaon-pion separation in the range $2$-$\SI{50}{\giga\eV}$}
    \begin{itemize}
      \item{Our studies focus mainly on flavour physics at the $Z$-pole}
    \end{itemize}
    \item{ARC will allow us to fully exploit the full range of flavour physics potential at future $e^+e^-$ colliders}
    \begin{itemize}
      \item{Will enhance the capabilities in Higgs, $WW$ and top physics}
    \end{itemize}
    \item{Next steps will include completing the optimisation, including magnetic field effects, and R\&D on photodetectors}
  \end{itemize}
  \begin{center}
    \huge Thank you for listening!
  \end{center}
\end{frame}

\end{document}
